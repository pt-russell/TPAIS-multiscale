\documentclass{article}
\usepackage{graphicx}
\usepackage{amsfonts}
\usepackage{url}
\usepackage{amsmath,amssymb}
\usepackage{enumerate}
\usepackage{color}
\usepackage{amsthm,amsmath}
\usepackage{bbm}
\usepackage{mhchem}

%for stacking below a sum or max...
\usepackage{mathtools}

\usepackage[colorlinks]{hyperref}

\newcommand{\comment}[2]{\vspace{0.6cm}{\bf Comment:} {\it #1.}

\vspace{0.3cm}{\bf Answer:} #2}

\begin{document}
\title{Response to Referee Comments: Transport map-accelerated adaptive importance sampling}
\maketitle

Firstly, we would like to thank the referees, associate editor and editor for their careful consideration of our manuscript. We largely agree with the comments made, and have adjusted the paper to reflect the extremely helpful suggestions that have been given. In particular, both referees commented on the long length of the paper and the somewhat disjoint feel between the two main topics discussed within it. In order to address this primary point, we have made substantial reductions to the paper in the areas suggested by the referees. We have also reduced the emphasis on the multiscale stochastic reaction network material, aiming to use this as an example of a nice application for the methods outlined, as opposed to an equally-weighted component of the work. We hope that this has gone a long way to knitting the two parts of the work together in a way which tells a more coherent story.

Below we list the comments made by the referees and respond to them in turn, highlighting where we have made changes in the manuscript to address them. We also attach a version of the paper with the changes in text highlighted in red. Note that a lot of the changes involved removal of material which is not visible in this document but is outlined in the responses below.

We believe that the manuscript is now tighter and improved, and we would like to thank the referees and editors for their time in considering our manuscript for publication in SIAM JUQ.


\section*{Referee 1 comments}

\comment{Length and scope. Not including references, this manuscript is 32 pages long and seems to have two somewhat disjoint parts. The first part develops the transport map accelerated importance sampler and provides details of the algorithm. The second part seems to focus more on multiscale approximations of stochastic reaction networks, and the impact that these approximations have on the posterior distribution. While interesting, especially within a model discrepancy context, I feel that this part of the paper is tangential to the paper's main algorithmic contributions. With the length of this paper in mind, I would suggest shortening Sections 6 and 7 in order to focus more on numerical comparisons of the new algorithm and less on stochastic reaction networks. Details of the multiscale material could be moved into supplementary material or even into another manuscript.}{We agree that the paper is lengthy, we shortened it considerably, and we welcome the suggestions as to how to make it more accessible. The method was designed with the application of inference of multiscale stochastic networks in mind, and the main examples in the manuscript concern this topic. However we agree that this is not the main focus of the paper, and so we have taken the referees suggestion onboard with the aim of reducing the amount of reaction network material. We are however keen to keep the results as shown in Figure 7.6 in the paper, since this was only possible with the aid of the methodology presented, but yet we do not believe it warrants a manuscript of its own. We have also rebalanced the introduction to make the main focus the MCMC methodology. Any required references for the material regarding multiscale methods for stochastic reaction networks are given in Section 6.

  The biggest change we have made following this comment is the heavy redaction of Section 6. This keeps the focus on the numerical methodology. The more detailed version of Section 6 is now included as supplementary material for interested readers.}

% I have reduced the introduction and it also flows better now.
%supplementary material done

\comment{Regularisation. In regards to the regularisation terms in equation (3.2), the authors claim that "experimentation has shown that the choice $\beta=1$ is sufficient for most problems." I would like to see more discussion on this choice. With a sufficiently large number of samples, decreasing $\beta$ should result in a more accurate transport map, so why does a smaller value of $\beta$ not yield a more efficient algorithm? With $\beta=1$, is the map somehow able to capture enough of the posterior structure to enable effective importance sampling?}{The regularisation term is present to prevent overfitting of the transport map to what is likely to be an incomplete MCMC sample. The stage at which the map is being constructed coincides with the burn-in of the ETAIS algorithm, and as such overfitting to incomplete data could produce a map which hinders further exploration of the state space. The value used has been shown to work well, as demonstrated in the numerical experiments presented. A discussion along these lines has been added after equation (3.2).}

%DONE

\comment{Proposal distribution. The proposed transport map ETAIS algorithm is tested using local random walk proposals for each ensemble member. However, for accurate maps, an independent Gaussian proposal or some form of mixture proposal could potentially generate nearly iid samples of the target distribution. Is there a reason for sticking with the random walk proposals? I think adding a comparison of proposal types, even if only on the Rosenbrock toy problem, would help readers understand the importance of each part in the transport map ETAIS algorithm: the proposal, map, resampler, and ensemble size.}{The referee is entirely correct that in the situation that the transport map is able to capture all of the posterior features, that independent sampling from the reference Gaussian will produce near iid samples from the density. However, as demonstrated in the examples in the paper, for example Figure 7.4, the pushforward distribution for complex examples is often quite far from the Gaussian reference distribution, which would lead to a poorly performing algorithm. The whole basis of combining the ETAIS approach with the transport maps is that for complex examples, the push forward density itself has complex structure which benefits from adaptive importance sampling. We have added a discussions on this topic to just before Section 5.2, just before 7.1.1, and also to the introduction, since this is a key justification for the approach.

The referee suggests a comparison of proposal types. This was done in detail for our first paper ``Ensemble Transport Adaptive Importance Sampling'', and was published in detail in Paul Russell's PhD thesis. We compared RW proposals with MALA and HMC. The performance of all of the proposals were similar, but the RW proposal is the cheapest. We have added a brief discussion about proposal types at the end of Section 2.}

%DONE

\comment{Ensemble Size. How is the ensemble size chosen? The authors mention that a smaller ensemble can be used when the transport map is accurate, but are there any guidelines on how to choose the ensemble size given the map?}{It is relatively easy to assess when the ensemble size is too small, since it leads to poor importance samples and larger variances in the weights. If the transport map is doing a good job, then the push forward distribution will be close to Gaussian, and then rules of thumb regarding the number of samples required to characterise an $n$-dimensional Gaussian can be used to suggest ensemble sizes. If variances of weights are very high, then it is likely that ensemble sizes will need to be increased. In general, the bigger the ensemble, the better the proposal, and the lower the variance of the resulting importance weights. However this has to be balanced with the increased costs of resampling, which grow polynomially in the ensemble size, depending on the method chosen. We have added a discussion on this to the end of Section 4.}

%DONE

\comment{Closing Remarks. I found the discussion (Section 8) ended rather abruptly. When reading the manuscript, I felt like I walked off the end of the manuscript. I think a sentence or two that ties things together would provide a better closing statement.}{We have added a paragraph to further explore our argument for combining AIS schemes and transport maps.}

%DONE

\comment{I would prefer numbering all equations, but leave that to the author's discretion.}{We had no strong opinion on this so are happy to go with this suggestion.}

%DONE

\comment{Subscript in Algorithm 1 on line 3. Should the proposal be centered at $\theta_j$ or $\theta_i$}{Corrected, thanks.}

%DONE


\comment{Second equation on page 4. Should the sample be labeled with a $\hat{ }$?}{Corrected, thanks.}

%DONE

\comment{When discussing monotonicity of the map, it might be worth pointing out the monotone parameterisations introduced in "An Introduction to Sampling via Measure Transport" by Marzouk et al.}{Thanks for this suggestion, we have added a comment and a reference.}

%DONE

\comment{In Section 4, the notation for proposed points on the reference space (marked with an apostrophe) and proposed points on the target space (marked with a $\hat{ }$) are different. It might be easier to follow if these were consistent, but that is up to the authors.}{We have done as the referee suggests, both are now marked with a $\hat{ }$.}
%DONE

\comment{Equation (6.2) $Y_j$ is used in the text but $X$ is used in the equation. Is that a typo?}{Corrected, thank you.}

%DONE

\comment{Sentence before equation (6.3). "variable" should be plural "variables"}{Corrected, thank you.}

%DONE

\section*{Referee 2 comments}

\comment{I do think that having these realistic applications is a strength—but I think the paper would be much easier to digest if the pieces were better woven together at the outset. On that note, the introduction takes a rather long time to get to the point of combining transport-ETIAS with partially observed stochastic reaction networks. Perhaps the necessary link here is “sloppiness,” mentioned on page 3?}{We agree with this comment. In particular, we have entirely removed the discussion on stochastic reaction networks from the introduction. This change moves the focus more firmly to that of the proposed sampling methods. As suggested by the referee we keep the discussion on sloppiness and unidentifiability, and briefly discuss multiscale systems as an example where this sloppiness often presents itself, leading to posterior densities which are highly concentrated near to lower-dimensional manifolds.}

% 1/2 DONE, still to do - move some of the olde introduction into the stochastic networks section

%Done now I think?

\comment{Also, it should be made more clear what exactly is new within the stochastic reaction networks portion of the paper}{Thanks to the referee for this comment. The main novel result in this portion of the paper is that summarised by the plots in Fig 7.4. The multiscale methods themselves have been presented in other work which was already heavily cited in this paper. I am not aware of anybody having formulated inverse problems using these methods before, but the treatment is natural. But the ability to sample from the complex posteriors arising from these problems, and therefore the ability to compare the posteriors arising from different multiscale approximations is new. Following this comment and comments from referee 1, we have heavily reduced the volume of material discussing the multiscale approximations in Section 6, which now appears in supplementary material. We have also heavily reduced the material in 7.1.1, with a simple presentation of Figure 7.4, the Bayes factors, and a short discussion.}

%DONE

\comment{In the numerical examples, what exactly is the “relative L2 error,” or $\delta_{L^2}$? I don’t see where this is defined, and I don’t know it as a standard MCMC or importance sampling diagnostic. Is it the relative $L_2$ error in the estimate of some posterior moment? If so, which moment? And relative to what? Similarly, can you define $\delta_{ESS}$ as a function of the ESS?}{The relative $L^2$ error in this context refers to the error made in approximating the PDF of the target density with a histogram of the samples. We have added in definitions for this. The two deltas, as described in the caption in the table, are the optimal scaling parameters, one optimised for ESS and the other for our relative $L^2$ measure. We have expanded on this in the caption, and hope that this is clearer now. Moreover, our notation wasn't consistent throughout the paper, as $\delta$ is actually the $\beta$ scaling parameter in the proposal distributions $q(\cdot;\cdot,\beta)$. This has now been corrected throughout.}

%DONE

\comment{Do the error versus \# samples plots (as in Figure 5.3, Figure 7.3, Figure 7.8) correspond to a single runs of MCMC? Or are they the error averaged over many independent chains (in
1
which case I think plotting all the runs in the same figure would be more illustrative). These are Monte Carlo schemes, so I expect the error in any given running estimate to be noisy.}{These plots are produced by averaging over 32 repeats. The plots show comparison of average convergence for several different algorithms. We tried creating the plots that the referee has suggested but the sheer number of lines led to plots which were in our opinion far less informative. We have added details of the averaging to the captions of all relevant plots.}

%DONE

\comment{By \# of samples, in the plots for MCMC error, do you simply mean the number of MCMC iterations? Similarly, what is meant by \# of samples in the ETIAS error plots? Using “samples” without context is confusing, as there are many intermediate populations of samples at hand.}{We literally meant the number of samples produced. For the MCMC methods we mean the number of iterations, since there is one sample per iteration. For ETAIS, we mean the number of iterations multiplied by the ensemble size. We have added a description along these lines below the first such example, Figure 5.3.}

%DONE

\comment{To complement these plots, I think that a very useful and widely used diagnostic, applicable both to MCMC and ETIAS and their transport-accelerated versions, would be ESS per posterior density evaluation, or simply ESS per unit of computational time.}{The ESS is a very commonly used and useful diagnostic, but we are of the opinion that it is flawed and can lead to misleading results, in particular in the scenarios we are considering where the posterior density is highly concentrated around a low dimensional manifold. The ESS for an MCMC sample which has sampled efficiently locally but failed to explore the whole of the state space, can give a deceptively high value. To avoid this, we have used a much more challenging metric, the relative $L^2$ error in the density, approximated on a uniform mesh. This is challenging both in terms of computing it, but it is also a challenging metric for the Monte Carlo methods, since it requires global convergence of the chains. We understand the reasons that the referee has asked for this metric to be considered, not least because it's use is ubiquitous, but we believe that the one we have used is actually a more rigorous measure.}

%DONE

\comment{I think there is some confusion about invertibility in various places.
In the discussion following Definition 3.2, it is stated that the space of all invertible maps might not contain an exact coupling. For absolutely continuous distributions with strictly positive densities, which seems to correspond to the cases here, I do not believe that this statement is correct. There is always a bijective map from one distribution to the other (and vice versa); indeed, there is an infinity of such maps. If one only considers maps with triangular structure, then there is a unique (up to sets of measure zero) invertible transformation. This is more or less the Knothe–Rosenblatt rearrangement; see one of Villani’s books, or [Bogachev, Kolesnikov, Medvedev 2005] for a more formal treatment.
Similarly, the statement in Section 8.2 about the true map being “invariably itself not invertible” seemed odd to me. If by the true map one means the KR rearrangement, then it is invertible. Maybe I am confused about what is meant by “true map?”}{Thanks to the referee for their comments here. We have amended the discussion following Definition 3.2, and we have removed Section 8.2.}

%DONE}

\comment{ In comparing the transport MH and the transport-ETIAS algorithms, I think it is somehow not surprising that in many problems the latter performs better. But I also have a hunch that the difference may be due to “global” versus local proposals. Transport MH, as I understand the implementation here, is restricted to using local random-walk proposals in the reference space. But some of the larger efficiency gains in [46] resulted from the use of a Metropolis independence proposal on the reference space, in particular draws from a standard Gaussian, i.e., the entire reference measure. These global moves yielded excellent mixing when the map was sufficiently accurate, and were combined with local moves using a delayed rejection scheme. ETIAS is not exactly the same, but it does have something in common with a global proposal. More broadly, this reflects the close link between independence Metropolis algorithms and importance sampling. Some comments (please agree or disagree) on this front would be illuminating.}{The first referee made a similar comment on this, which we hope addresses this comment also. We also refer the referee to the discussion we have added on this in the manuscript just before Section 5.2, just before 7.1.1, and also to the introduction. The problems we are attempting to sample from in this work are highly challenging, with the density highly concentrated on a low dimensional curved manifold, and in the majority of cases the low dimensional approximations of the transport map are not sufficient to make independent sampling efficient.}

%DONE

\comment{What part of Section 6 is newly developed in this paper? Is the derivation of the exact likelihood in Section 6.1 new? Or is the approximation in Section 6.2 new? This relates to my broader comment above. It would be good to have a clearer picture of where exactly this paper’s key contributions lie, and even of their relative weight.}{We have not seen the derivation of the likelihood anywhere, but it is a natural treatment of the likelihood of a continuous time Markov chain, and so following comments from both referees we have removed this from the paper. It now appears in the supplementary material for interested readers.}

%DONE

\comment{Here are several suggestions for content that could be removed entirely or shortened, along with a few areas that could be expanded:}{Thanks for these suggestions, we will address each in turn below.}

%DONE

\comment{Suggestion for removal: The use of log-transformations to map strictly positive variables to the real line is a standard trick in MCMC, and has also been employed in previous work on transport. One could simply mention this and move forward. Thus I don’t see the need of including any results without the log preconditioner in problems where it is warranted. Log- preconditioning is well recognized as a good thing to do.}{Thanks to the referee for these very helpful suggestions. We agree that the log transformation is pretty standard, and so we have vastly reduced the discussion and numerical experiments presented on this topic to just a single comparison.}

%DONE

\comment{Suggestion for removal: A lot of Section 3 could be abridged by appropriate referencing of [46].}{
  We have removed as much as possible, but we wanted the paper to remain as standalone as possible to make it more accessible and easier to read, and also to highlight the differences in finding in the map with importance samples. We have removed quite a bit, in particular Section 3.2.}

%DONE

\comment{Suggestion for removal: Section 4 is quite terse, but the“second option” (Algorithm 3)—both proposing and resampling on the reference space—seems to be the cleaner algorithm and, if I understand the authors, the more effective one. So perhaps the first option can be mentioned in passing, and all the results can focus on the second? Is there any good reason to use Algorithm 2 instead of Algorithm 3?}{
We now only briefly describe algorithm 2, and no longer give pseudocode for this option. We have kept the numerics in which the results are compared and a brief discussion.}

%DONE

\comment{Suggestion for removal: The description of the ETIAS algorithm in Section 2 is very nice, but here an illustrative figure might help.}{There is a nice illustration in a paper by Bugallo et al which demonstrates how AIS works, so rather than making a poor reproduction, we have cited this paper.}

%DONE

\comment{End of Section 2, “learn local covariances across the whole of the domain”: I believe I under- stand what is meant, but this awkwardly phrased}{Thanks, we have rewritten this.}

%DONE

\end{document}
