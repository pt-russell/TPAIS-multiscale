\documentclass[final]{siamltex}
%test change
\usepackage{cite}
\usepackage{graphicx,bbm,pstricks,soul}
\usepackage{pifont}
\usepackage{bbm,algorithmic,mdframed,placeins,multirow,booktabs,subfigure}
\usepackage[ruled,vlined,linesnumbered]{algorithm2e}

\setlength{\parindent}{0in}
\usepackage{amsmath,amsfonts,amsbsy,amssymb}
\newcommand{\RARR}[3]{#1
  \;\displaystyle\mathop{\displaystyle\longrightarrow}^{#3}\; #2}
\newcommand{\RARRlong}[3]{#1
  \;\displaystyle\mathop{-\!\!\!-\!\!\!-\!\!\!-\!\!\!-\!\!\!\!\displaystyle
  \longrightarrow}^{#3}\; #2}
\newcommand{\LARR}[3]{#1
  \;\displaystyle\mathop{\displaystyle\longleftarrow}^{#3}\; #2}
\newcommand{\LRARR}[4]{{\mbox{ \raise 0.4 mm \hbox{$#1$}}} \;
  \mathop{\stackrel{\displaystyle\longrightarrow}\longleftarrow}^{#3}_{#4}
  \; {\mbox{\raise 0.4 mm\hbox{$#2$}}}}
\newcommand{\bX}{{\bf X}}
\newcommand{\vecx}{{\mathbf x}}
\newcommand{\vecy}{{\mathbf y}}
\newcommand{\vecz}{{\mathbf z}}
\newcommand{\vecq}{{\mathbf q}}
\newcommand{\bs}{{\mathbf s}}
\newcommand{\vecr}{{\mathbf r}}
\newcommand{\vecX}{{\mathbf X}}
\newcommand{\vecv}{{\mathbf v}}
\newcommand{\tick}{\ding{52}}
\newcommand{\cross}{\ding{54}}
\newcommand{\vecn}{{\mathbf n}}
\newcommand{\vecp}{{\mathbf p}}
\newcommand{\cT}{{\mathcal T}}
\newcommand{\dt}{{\mbox{d}t}}
\newcommand{\dx}{{\mbox{d} \vecx}}
\newcommand{\boldnu}{{\boldsymbol \nu}}
\newcommand{\er}{{\mathbb R}}
\renewcommand{\div}{{\rm div}}
\newcommand{\bnu}{{\bf \nu}}
\newcommand{\divergence}{\mathop{\mbox{div}}}
\renewcommand{\P}{\mathbb{P}}
\newcommand{\FP}{P_{\rm{FP}}}
\newcommand{\ME}{P_{\rm{ME}}}
\newcommand{\MEs}{P_{\rm{ME}_{S}}}
\newcommand{\D}{\mathcal{D}}
\newcommand{\G}{\mathcal{G}}
\newcommand{\N}{\mathcal{N}}
\newcommand{\X}{{\mathbf X}}
\newcommand{\Y}{{\mathbf Y}}
\newcommand{\W}{{\mathbf W}}
\newcommand{\data}{D}
\newcommand{\neff}{n_{\text{eff}}}
\newcommand{\E}{{\mathbb E}}
\renewcommand{\b}[1]{{\bf #1}}
\DeclareMathOperator*{\argmin}{arg\,min}
\DeclareMathOperator*{\argmax}{arg\,max}

\newcommand{\picturesAB}[6]{
\centerline{
\hskip #4
\raise #3 \hbox{\raise 0.9mm \hbox{(a)}}
\hskip #5
\epsfig{file=#1,height=#3}
\hskip #6
\raise #3 \hbox{\raise 0.9mm \hbox{(b)}}
\hskip #5
\epsfig{file=#2,height=#3}
}}
\newcommand{\picturesCD}[6]{
\centerline{
\hskip #4
\raise #3 \hbox{\raise 0.9mm \hbox{(c)}}
\hskip #5
\epsfig{file=#1,height=#3}
\hskip #6
\raise #3 \hbox{\raise 0.9mm \hbox{(d)}}
\hskip #5
\epsfig{file=#2,height=#3}
}}

\makeatletter  
\newcommand{\xleftrightarrows}[2][]{\mathrel{%  
 \raise.40ex\hbox{$  
       \ext@arrow 3095\leftarrowfill@{\phantom{#1}}{#2}$}%  
 \setbox0=\hbox{$\ext@arrow 0359\rightarrowfill@{#1}{\phantom{#2}}$}%  
 \kern-\wd0 \lower.4ex\box0}}  
 
\newcommand{\xrightleftarrows}[2][]{\mathrel{%  
 \raise.40ex\hbox{$\ext@arrow 3095\rightarrowfill@{\phantom{#1}}{#2}$}%  
 \setbox0=\hbox{$\ext@arrow 0359\leftarrowfill@{#1}{\phantom{#2}}$}%  
 \kern-\wd0 \lower.4ex\box0}}  
 
\def\leftrightarrowfill@{%
 \arrowfill@\leftarrow\relbar\rightarrow%
 }
\newcommand*{\centerfloat}{%
  \parindent \z@
  \leftskip \z@ \@plus 1fil \@minus \textwidth
  \rightskip\leftskip
  \parfillskip \z@skip}
\makeatother
\makeatother 

\title{Transport map accelerated-PAIS, and application to inverse problems arising from
  multiscale stochastic reaction networks}
\author{Simon Cotter, Yannis Kevrekidis, Paul Russell}
\begin{document}
\maketitle
\begin{abstract}
In many applications, inverse problems arise where where there are
complex correlations between the different parameters which we wish to
infer from data. The correlations often manifest themselves as lower
dimensional manifolds on which the likelihood function is
invariant, or varies very little. This can be due to trying to infer
unobservable parameters, or due to sloppiness in the model which is
being used to describe the data. In such a situation, standard
sampling methods for characterising the posterior distribution which
do not incorporate information about this structure will be highly
inefficient. Moreover, most methods are inherently serial in nature,
and as such are not expoiting the parallelised  nature of modern
computer infrastructure. In this paper, we seek to develop a method to
tackle this problem, using optimal transport maps to simplify
posterior distributions which are concentrated on lower dimensional
manifolds.

We demonstrate the approach by considering inverse problems arising
from partially observed stochastic reaction networks. In particular,
we consider systems which exhibit multiscale behaviour, but for which
only the slow variables in the system are observable. We demonstrate
that certain multiscale approximations lead to more consistent
approximations of the posterior than others.
\end{abstract}


\section{Introduction}
%Topics to cover: parallel MCMC, pMC, PAIS
%Transport maps, transport map MCMC
%Stochastic reaction networks
%Multiscale approximations, QSSA/QEA, CMA

In Section \ref{sec:map} we show how an appropriate transport map can
be constructed from importance samples which maps the posterior close
to a reference Gaussian measure. In Section \ref{sec:TPAIS} we show
how such a map can be incorporated into a sophisticated parallel MCMC
infrastructure in order to accelerate mixing. In Section
\ref{sec:multi} we consider how likelihoods can be approximated using
multiscale methodologies in order to carry out inference for
multiscale and/or partially observed stochastic reaction networks. In
Section \ref{sec:num} we present some numerical examples, which serve
to demonstrate the increased efficiency of the described sampling
methodologies, as well as investigating the posterior approximations
discussed in the previous section. We conclude with a discussion in
Section \ref{sec:conc}.

\section{Construction of transport maps in importance sampling} \label{sec:map}
%Follow the other papers, show adaptation(s)

\section{T-PAIS}\label{sec:TPAIS}
%Plug into PAIS

\section{Multiscale approximations of likelihoods in stochastic
  reaction networks}\label{sec:multi}

\section{Numerical Examples}\label{sec:num}

\section{Conclusions}\label{sec:conc}

\bibliographystyle{siam}
\bibliography{refs}
\end{document}
